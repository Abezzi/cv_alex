\documentclass[11pt]{article}       % set main text size
\usepackage[letterpaper,                % set paper size to letterpaper. change to a4paper for resumes outside of North America
top=0.5in,                          % specify top page margin
bottom=0.5in,                       % specify bottom page margin
left=0.5in,                         % specify left page margin
right=0.5in]{geometry}              % specify right page margin
                       
\usepackage{XCharter}               % set font. comment this line out if you want to use the default LaTeX font Computer Modern
\usepackage[T1]{fontenc}            % output encoding
\usepackage[utf8]{inputenc}         % input encoding
\usepackage{enumitem}               % enable lists for bullet points: itemize and \item
\usepackage[hidelinks]{hyperref}    % format hyperlinks
\usepackage{titlesec}               % enable section title customization
\raggedright                        % disable text justification
\pagestyle{empty}                   % disable page numbering

% ensure PDF output will be all-Unicode and machine-readable
\input{glyphtounicode}
\pdfgentounicode=1

% format section headings: bolding, size, white space above and below
\titleformat{\section}{\bfseries\large}{}{0pt}{}[\vspace{1pt}\titlerule\vspace{-6.5pt}]

% format bullet points: size, white space above and below, white space between bullets
\renewcommand\labelitemi{$\vcenter{\hbox{\small$\bullet$}}$}
\setlist[itemize]{itemsep=-2pt, leftmargin=12pt, topsep=7pt} %%% Test various topsep values to fix vertical spacing errors

% resume starts here
\begin{document}

% name
\centerline{\Huge Alex Herrera Saravia}

\vspace{5pt}

% contact information
\centerline{Talca, Regi\'on del Maule, Chile}
\centerline{\href{mailto:abezzi@outlook.cl}{abezzi@outlook.cl} | +569 8616 5263 | \href{https://alexherrera.cl}{alexherrera.cl} | \href{https://github.com/Abezzi}{github.com/abezzi} | \href{https://linkedin.com/in/alex-herrera-saravia/}{linkedin.com/in/alex-herrera-saravia}}

\vspace{-10pt}

% skills section
\section*{Habilidades}
Python, Django, React, TypesCript, Redux Toolkit, TailwindCSS, VueJs, NodeJS, PHP, Docker, PostgreSQL, MySQL, Pruebas Unitarias, JavaScript, HTML, CSS, bootstrap, TypeORM, Soluci\'on de problemas, Estructuras de Datos, Linux, Metodolog\'ias \'Agiles, Vim, Netlify,  \\

\vspace{-6.5pt}

% experience section
\section*{Experiencia}
\textbf{Desarrollador de Software,} {OneConsultores SpA} -- Talca, Regi\'on del Maule, Chile \hfill Octubre 2021 -- Mayo 2023 \\
\vspace{-9pt}
\begin{itemize}
  \item Desarrollé el módulo de juzgado el cual requería digitalizar la gestión de trámites judiciales, por lo que tomé los requerimientos y diseñé un sistema de acceso virtual con VueJS, TypeORM, NodeJS y PostgreSQL, que permitía a personas naturales consultar el estado de sus trámites y a funcionarios registrar información, documentos y resoluciones judiciales, logrando un aumento en la velocidad de los procesos, mayor comodidad para los usuarios y la integridad de los datos mediante respaldo digital.
  \item Optimicé el sistema en producción de permisos de circulación que requería mayor estabilidad y funcionalidades, asumiendo la tarea de implementar mejoras, como la visualización de documentos desde el formulario de ingreso, la migración del gestor de correos en y la corrección de errores con VueJS y KumbiaPHP, consiguiendo un sistema más estable, eficiente y con una experiencia de usuario mejorada.
  \item Diseñé y desarrollé una aplicación móvil en IONIC para gestionar el ingreso al evento FIAC 2023, enfrentando el desafío de garantizar un control eficiente y seguro de los asistentes; mi tarea consistió en crear una solución que permitiera escanear el código de la cédula de identidad y registrar la información en una base de datos a través de una API, facilitó la gestión del evento para los organizadores.
  \item Resolví incidencias y optimicé la experiencia de usuarios en un sistema municipal al brindar soporte técnico especializado, orientando a personas con dificultades para utilizar diversas funcionalidades del sistema; mi tarea incluyó generar "informes de transparencia" los cuales eran solicitados por ciudadanos. Logré mejorar la satisfacción de los usuarios y garantizar el acceso transparente a la información pública, fortaleciendo la confianza en el sistema.
\end{itemize}

\textbf{Practicante} {OneConsultores SpA} -- Talca, Regi\'on del Maule, Chile \hfill Enero 2021 -- Marzo 2021 \\
\vspace{-9pt}
\begin{itemize}
  \item Modernicé un sistema legacy actualizando mantenedores con VueJS 3, NodeJS, TypeORM y PostgreSQL, integrando webservices para optimizar el sistema de permisos de circulación; implementé mejoras en vistas y funcionalidades, creando interfaces intuitivas, lo que resultó en un sistema más rápido, estable y fácil de usar, reduciendo tiempos de procesamiento y facilitando la gestión de permisos para funcionarios municipales.
\end{itemize}

\section*{Participación Adicional}
\textbf{Competidor} {ACM-ICP latin american programming contest} \hfill Noviembre 2017 \\
\textbf{Competidor} {ACM-ICP latin american programming contest} \hfill Noviembre 2016 \\
\vspace{-9pt}

% projects section
\section*{Proyectos}
\textbf{Restaurant Full-Stack} \hfill \href{https://github.com/Abezzi/restaurant-fullstack/}{github.com/abezzi/restaurant-fullstack} \\
\vspace{-9pt}
\begin{itemize}
  \item Sistema para la administración de restaurantes o servicios similares, algunas de las características que contiene son: monorepo, punto de venta, perfiles, inicio de sesión mediante token, i18n, throttle, perfiles para la autoridad, navegación en base a roles, modo nocturno, configuración y personalización de la empresa. Las tecnologías utilizadas empezando por el front-end son: React, TypeScript, Vite, TailwindCSS y Redux Toolkit. Y en la contraparte de back-end: Django, DRF, Djoser, PostgreSQL, Git y pruebas unitarias.
\end{itemize}

\vspace{-9pt}

% education section
\section*{Educación}
\textbf{Universidad Cat\'olica del Maule} -- Ingeniería Ejecución Informática \hfill \\

\end{document}